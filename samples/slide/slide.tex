\documentclass[12pt,dvipdfmx]{beamer}
\usetheme{metropolis}
%\documentclass[slide,25pt,papersize]{jsarticle}
\usepackage{otf}
\usepackage{minijs}
\renewcommand{\kanjifamilydefault}{\gtdefault}
\usepackage{amsmath,amssymb,amsthm}
\usepackage[dvipdfmx]{}
\usepackage{fancybox}
\usepackage{ascmac}
%\usepackage{fancyhdr}
%\usepackage{lastpage}
\usepackage{listings}
\usepackage{physics}
\usepackage{bm}
\usepackage{ulem}
\usepackage{color}
\usepackage{tikz}
\usepackage[absolute,overlay]{textpos}
%\usepackage[colorgrid,gridunit=pt,texcoord]{eso-pic}
\usefonttheme{professionalfonts}
\usetikzlibrary{positioning,calc,shapes,snakes,shapes.callouts,shapes.arrows,decorations,decorations.pathmorphing}
\usepackage{tcolorbox}
\tcbuselibrary{theorems}
\tcbuselibrary{theorems,skins}
\usepackage{setspace}
%\usepackage{fancyhdr}

\lstset{
 	breaklines = true,
 	breakindent = 10pt,
 	basicstyle = \ttfamily\scriptsize,
 	classoffset = 0,
 	%枠 "t"は上に線を記載, "T"は上に二重線を記載
	%他オプション:leftline,topline,bottomline,lines,single,shadowbox
 	frame = tbrl,
 	%frameまでの間隔(行番号とプログラムの間)
 	framesep = 5pt,
 	%行番号の位置
 	numbers = left,
	%行番号の間隔
 	stepnumber = 1,
	%行番号の書体
 	numberstyle = \tiny,
	%タブの大きさ
 	tabsize = 4,
 	%キャプションの場所("tb"ならば上下両方に記載)
 	captionpos = t
}

\title{title}
\author{author}
\date{date}
\institute{institute}

\begin{document}
\thispagestyle{empty}
\frame{\maketitle}


\begingroup 
	\setbeamertemplate{frametitle}{}
	%\setbeamertemplate{footline}{}
	\setbeamertemplate{navigation symbols}{}
	\setbeamercolor{background canvas}{bg=green!50!white}

	\tikzset{
		fuki/.style={
			fill=black!20!white,
			overlay,
			rectangle callout,
			rounded corners=3pt,
			inner sep=10pt
		}
	}
	\begin{frame}{blank page}
		\only<2->{\begin{textblock*}{70pt}(0.5\textwidth,60pt)
			\tikz\node[fuki,callout absolute pointer={(-60pt,5pt)}]{\begin{minipage}{\textwidth}
			\fontsize{20pt}{0pt}\selectfont
			\centering
			hello
			\end{minipage}};\end{textblock*}}
		\only<3->{\begin{textblock*}{20pt}(180pt,200pt)
			\fontsize{20pt}{0pt}\selectfont
			\raisebox{-12pt}{\tcbox[
        enhanced,         % これは必須
        frame hidden,     % 枠を消す. これも enhanced 必須
        colback=cyan!50!yellow,
			]{world}}
		\end{textblock*}}
	\end{frame}
\endgroup
\end{document}

